% Options for packages loaded elsewhere
\PassOptionsToPackage{unicode}{hyperref}
\PassOptionsToPackage{hyphens}{url}
\PassOptionsToPackage{dvipsnames,svgnames,x11names}{xcolor}
%
\documentclass[
  letterpaper,
  DIV=11,
  numbers=noendperiod]{scrartcl}

\usepackage{amsmath,amssymb}
\usepackage{iftex}
\ifPDFTeX
  \usepackage[T1]{fontenc}
  \usepackage[utf8]{inputenc}
  \usepackage{textcomp} % provide euro and other symbols
\else % if luatex or xetex
  \usepackage{unicode-math}
  \defaultfontfeatures{Scale=MatchLowercase}
  \defaultfontfeatures[\rmfamily]{Ligatures=TeX,Scale=1}
\fi
\usepackage[]{libertinus}
\ifPDFTeX\else  
    % xetex/luatex font selection
\fi
% Use upquote if available, for straight quotes in verbatim environments
\IfFileExists{upquote.sty}{\usepackage{upquote}}{}
\IfFileExists{microtype.sty}{% use microtype if available
  \usepackage[]{microtype}
  \UseMicrotypeSet[protrusion]{basicmath} % disable protrusion for tt fonts
}{}
\makeatletter
\@ifundefined{KOMAClassName}{% if non-KOMA class
  \IfFileExists{parskip.sty}{%
    \usepackage{parskip}
  }{% else
    \setlength{\parindent}{0pt}
    \setlength{\parskip}{6pt plus 2pt minus 1pt}}
}{% if KOMA class
  \KOMAoptions{parskip=half}}
\makeatother
\usepackage{xcolor}
\usepackage[nomarginpar,top=0.5in,bottom=0.5in,outer=0.5in,inner=0.5in]{geometry}
\setlength{\emergencystretch}{3em} % prevent overfull lines
\setcounter{secnumdepth}{-\maxdimen} % remove section numbering
% Make \paragraph and \subparagraph free-standing
\ifx\paragraph\undefined\else
  \let\oldparagraph\paragraph
  \renewcommand{\paragraph}[1]{\oldparagraph{#1}\mbox{}}
\fi
\ifx\subparagraph\undefined\else
  \let\oldsubparagraph\subparagraph
  \renewcommand{\subparagraph}[1]{\oldsubparagraph{#1}\mbox{}}
\fi


\providecommand{\tightlist}{%
  \setlength{\itemsep}{0pt}\setlength{\parskip}{0pt}}\usepackage{longtable,booktabs,array}
\usepackage{calc} % for calculating minipage widths
% Correct order of tables after \paragraph or \subparagraph
\usepackage{etoolbox}
\makeatletter
\patchcmd\longtable{\par}{\if@noskipsec\mbox{}\fi\par}{}{}
\makeatother
% Allow footnotes in longtable head/foot
\IfFileExists{footnotehyper.sty}{\usepackage{footnotehyper}}{\usepackage{footnote}}
\makesavenoteenv{longtable}
\usepackage{graphicx}
\makeatletter
\def\maxwidth{\ifdim\Gin@nat@width>\linewidth\linewidth\else\Gin@nat@width\fi}
\def\maxheight{\ifdim\Gin@nat@height>\textheight\textheight\else\Gin@nat@height\fi}
\makeatother
% Scale images if necessary, so that they will not overflow the page
% margins by default, and it is still possible to overwrite the defaults
% using explicit options in \includegraphics[width, height, ...]{}
\setkeys{Gin}{width=\maxwidth,height=\maxheight,keepaspectratio}
% Set default figure placement to htbp
\makeatletter
\def\fps@figure{htbp}
\makeatother

\KOMAoption{captions}{tableheading}
\makeatletter
\@ifpackageloaded{caption}{}{\usepackage{caption}}
\AtBeginDocument{%
\ifdefined\contentsname
  \renewcommand*\contentsname{Table of contents}
\else
  \newcommand\contentsname{Table of contents}
\fi
\ifdefined\listfigurename
  \renewcommand*\listfigurename{List of Figures}
\else
  \newcommand\listfigurename{List of Figures}
\fi
\ifdefined\listtablename
  \renewcommand*\listtablename{List of Tables}
\else
  \newcommand\listtablename{List of Tables}
\fi
\ifdefined\figurename
  \renewcommand*\figurename{Figure}
\else
  \newcommand\figurename{Figure}
\fi
\ifdefined\tablename
  \renewcommand*\tablename{Table}
\else
  \newcommand\tablename{Table}
\fi
}
\@ifpackageloaded{float}{}{\usepackage{float}}
\floatstyle{ruled}
\@ifundefined{c@chapter}{\newfloat{codelisting}{h}{lop}}{\newfloat{codelisting}{h}{lop}[chapter]}
\floatname{codelisting}{Listing}
\newcommand*\listoflistings{\listof{codelisting}{List of Listings}}
\makeatother
\makeatletter
\makeatother
\makeatletter
\@ifpackageloaded{caption}{}{\usepackage{caption}}
\@ifpackageloaded{subcaption}{}{\usepackage{subcaption}}
\makeatother
\ifLuaTeX
  \usepackage{selnolig}  % disable illegal ligatures
\fi
\usepackage{bookmark}

\IfFileExists{xurl.sty}{\usepackage{xurl}}{} % add URL line breaks if available
\urlstyle{same} % disable monospaced font for URLs
\hypersetup{
  pdftitle={Rules},
  colorlinks=true,
  linkcolor={blue},
  filecolor={Maroon},
  citecolor={Blue},
  urlcolor={Blue},
  pdfcreator={LaTeX via pandoc}}

\title{Rules}
\author{}
\date{}

\begin{document}
\maketitle

\section{Core Rules}\label{core-rules}

\subsection{Characters}\label{characters}

A basic player character statblock looks like this:

Health: (3)9\\
Movement Speed: 4

STR: 0\\
AGI: 0\\
WIL: 0\\
PER: 0\\
SOC: 0

\subsubsection{Health}\label{health}

Character health is tracked as (X)Y where X is critical or ``crit''
health and Y is standard health. When crit health reaches 0, the
character falls unconscious and begins dying. When a character runs out
of standard health all subsequent damage reduces crit health. Crit
damage reduces crit health directly, it does not matter how much
standard health a character still has. Crit damage is notated with
parentheses.

This health score of (3)9 is higher than what most people have! a very
old or frail person might be (1)1, a normal adult might be (2)6. The
default character block with (3)9 health and 0s for all attributes
represents a healthy, able person with no deficiencies.

\subsubsection{Movement Speed}\label{movement-speed}

``Movement Speed'' determines how many spaces a character moves when
they spend 1 AP to move during initaitive.

\subsubsection{Attributes}\label{attributes}

Attribute scores are not mandatory descriptions of qualities. They
represent abilities or tools that the character has. They tend to range
from -3 to +3. The attribute score is the same as the modifier on a
check using that attribute. The ``passive'' score, the score that other
characters roll against, is 6+SCORE. If you do not have an attribute
score, it is not a narrative tool that you use to solve problems. You
can have attribute scores not listed here.

This set of scores is listed because they are the ``defensive''
attributes. That means these are the only attributes that other
characters are allowed to roll checks against. If you do not have an
attribute score, any checks made against that attribute succeed
automatically.

\paragraph{STR / Strength}\label{str-strength}

Strength is used to perform actions like lifting or breaking things as
well as some actions like grappling. It's also used to resist physical
conditions like poison, disease, and exhaustion.

\paragraph{AGI / Agility}\label{agi-agility}

AGI is your ability to evade attacks, grab on to ledges to keep yourself
from falling, avoid falling rocks, etc. It is your ability to avoid
physical danger.

It is also your ability to do a flip, jump over a chasm, etc. This is
not your DEX / Dexterity! DEX is your fine motor skills, your ability to
do card tricks, pickpocket, and juggle. DEX is not a ``defensive''
score.

\subparagraph{Attack Checks}\label{attack-checks}

Attacks are checks against passive AGI. The attacker wins ties. Some
mechanics like armor can replace passive AGI in the context of attack
checks.

\paragraph{WIL / Will}\label{wil-will}

WIL is your ability to resist mental conditions like fear. It is used to
resist many supernatural effects. It is also used for most supernatural
abilities, like writing hexes.

\paragraph{PER / Perception}\label{per-perception}

PER is your resistance to stealth and being tricked.

It is also your ability to investigate, spot things other people miss,
quickly find things, etc.

\paragraph{SOC / Social}\label{soc-social}

SOC is your resistance to being manipulated.

SOC is also your ability to manipulate, to schmooze. It's your gab, your
social acumen. Social rules are covered in greater detail later in this
document.

\subsubsection{Checks and Contests}\label{checks-and-contests}

When a character wants to do something using an attribute they roll 2d6
and add the relevant SCORE. The result of the roll is determined by
comparing the roll to a Difficulty Class or DC. This is a ``check''. If
the character's check is the same as the DC or higher, they succeed.
Often the DC for a check is the passive SCORE of another character.

If two charactes are in direct competition they are in a ``contest''.
They both roll 2d6 and add their scores. The higher score wins. In the
case of a tie, they either tie in the game or if that does not make
sense they redo the contest.

\paragraph{Additional Effects on
Success}\label{additional-effects-on-success}

Sometimes, skills or weapons will have additional effects triggered ``on
6'', ``on at least one 6'' or ``on matching''. This means that if you
succeed and the dice you rolled fulfilled the requirement (being a six,
both dice matching, etc.) the effect is activated.

``on 6'' effects can be triggered twice if you roll two 6s, ``at least
one 6'' effects only trigger once. ``on matching'' triggers on any set
of matching dice rolled.

\subsubsection{Advantage / Disadvantage}\label{advantage-disadvantage}

When you have ``advantage'', roll an extra d6 for your check and pick
two. When you have ``disadvantage'', roll another d6, then pick the two
lowest rolls. Advantage and disadvantage cancel each other out and are
cumulative.

If you somehow have disadvantage twice and advantage four times, you
have advantage twice. Roll four dice and pick two.

\subsubsection{Initiative}\label{initiative}

AP is used during ``initiative''. Initiative is any situation where many
characters want to do things at the same time and the order in which
things happen is important. Initiative is most often combat, but it can
include other time sensitive events like chases or races.

Every character has 3 AP cards.5 The cards say ``1 AP'' and indicate
what character they belong to with a name, a symbol, or both. Most
actions cost 1 AP, but you can do a ``reasonable amount'' of simple
things like talking for free.

Some terms:

\textbf{Deck}: This is the deck of cards that the DM (or whoever) draws
from during initiative. The deck that the whole table uses.

\textbf{Discard}: The discard pile for the whole table.

\textbf{Hand}: The cards that you have been dealt and are holding on to.

\textbf{``Spending'' AP}: Cards that have been used or removed that go
to the discard and are shuffled back together for the next round's deck.

\textbf{``Losing'' AP}: ``Lost'' AP is not shuffled back into the deck
at the end of the round. It must be ``regained'' either mechanically or
automatically at the end of initiative.

\textbf{``Interrupt''}: Spend 1 AP, act immediately

When initiative starts all characters pass their cards to the DM. This
includes their AP and any other special cards. The DM shuffles these
cards together to create the deck. The DM pulls a card from the top of
the deck and either deals the card to the character/player that it
belongs to or follows the rules on the card.

When a character is dealt a card they can choose to act using whatever
cards they currently have in their hand. They can also just hold on to
the card. When cards are spent or activated they get placed in the
discard.

Any character can interrupt in order to act at any time, doing this
costs 1 AP. The round ends when the table deck is empty and everyone who
still has AP has had a chance to interrupt to act with their remaining
AP.

At the end of a round, cards that have been discarded are shuffled
together to create the deck for the next round. Cards that have been
lost are not, they are set aside until they are regained, or initiative
ends.

This initiative system is intended for combat that is fast and
unpredictable.

\paragraph{Multiple -\textgreater{} AP -\textgreater{}
Costs}\label{multiple---ap---costs}

Some types of actions have multiple AP costs. The first time you perform
the action in a round it costs the first amount, the second time you do
it, it costs the second, etc. These costs are connected with arrows,
like this: 1-\textgreater2-\textgreater3. This does not limit how many
times you can perform an action. If you are at the last cost, pay that
amount of AP again to do the action.

\paragraph{Temp AP and Regaining AP}\label{temp-ap-and-regaining-ap}

``Temp AP'' must be spent immediately when it is recieved or it is lost.
If you spend no AP for an entire round, regain 1 AP.

\paragraph{``Beginning of the Round''}\label{beginning-of-the-round}

Very rarely, the particular order things happen in the ``beginning of
the round'' will matter. In that case, effects resolve in this order:

\subparagraph{First, Effect Management}\label{first-effect-management}

Any already active effects that trigger ``at the beginning of a round''
happen first. It happens before anything else in the round does.
Examples includes taking damage from {[}poison{]} or {[}bleed{]} or
gaining health from {[}heal{]}. If you have some effect or condition
that causes you to do some sort of roll or contest, do that now.

\subparagraph{Next, Status Management}\label{next-status-management}

Statuses are added or removed now. A point of {[}poison{]} is removed, a
point of {[}heal{]} is removed, anything that would end at the
``beginning'' of this round ends now.

\subparagraph{Skills}\label{skills}

Some moves and skills can be used ``at the beginning of the round''.
These must be used after effects and statuses resolve.

\subsubsection{Character Advancement and
XP}\label{character-advancement-and-xp}

Characters gain XP by solving problems, navigating difficult situatoins,
etc. XP can be used to increase attribute scores or acquire more skills.
There are no character levels in Actlite.

\paragraph{Increasing Attribute
Scores}\label{increasing-attribute-scores}

Increasing a negative attribute costs 1 XP. Increasing a positive
attribute costs a number of XP equal to the new attribute score. This
means going from 2 STR to 3 STR costs 3 XP. Going from -1 STR to 2 STR
costs \textbf{1} (-1 to 0) \textbf{+ 1} (0 to 1) \textbf{+ 2} (1 to 2)
\textbf{= 4} XP.

Some species or groups have hard limits on attribute scores and/or
alternative costs for upgrading some attribute scores.

\paragraph{Archetype Tables}\label{archetype-tables}

Characters gain skills by rolling on ``Archetype Tables'', collections
of thematically or mechanically similar skills. Archetype Tables can
have attribute score requirements or some other setting-specific
requirement. They can also have varying XP costs for rolls, notated
similarly to variable AP costs so that the first roll costs X, the
second roll costs Y, and all subsequent rolls have an XP cost of Z,
notated X-\textgreater Y-\textgreater Z. This allows some tables to
reward specialization, and some to punish it. If no cost is listed,
rolling costs on the table costs 1 XP.

Skills in tables are loosely ordered by desirability. If a player gets a
skill they already have by rolling on an archetype table, they get the
skill below/before it in the table. If none are below it, they get the
first skill above it. A player cannot roll on a table when they have
every skill on the table.

\subsection{Misc. Rules}\label{misc.-rules}

\subsubsection{Searching and Switching During
Combat}\label{searching-and-switching-during-combat}

Drawing, stowing, switching weapons, or picking a weapon up off the
ground costs 1 AP, dropping a weapon is free. Searching for something
you have on your person but you do not have ready is 1 AP. If what you
need is further away, like in a backpack you are wearing, pay 2 AP to
take off the bag, open it, rummage through it, etc.

\subsection{Stealth}\label{stealth}

Stealth is an AGI check against the defender's passive perception. It's
a check against a passive score, and that is always 6 + SCORE.

\subsubsection{Grappling}\label{grappling}

To start a grapple, perform a ``grapple check'', which is a +STR to-hit
{[}reach: close{]} attack that costs 2 AP. While grappling an opponent,
you have advantage on {[}reach: close{]} attacks against them, they have
disadvantage on attacks against you. You can move them by one space or
you can both go to the ground at the cost of 1 AP. Defender can spend 2
AP to trigger another STR contest to try and break free. Defender can do
nothing but attack and try to end the grapple. You cannot grapple a
creature larger than yourself. There are grappling skills that add a lot
of other options.

\subsubsection{Fall Damage}\label{fall-damage}

Fall damage is a d4 for every 6 spaces up to 25d4.

\section{Weapons}\label{weapons}

Here's a basic weapon, a club:

Club Tags: pole, one-handed Speed: 1-\textgreater2 Attacks: - 1d4 B, On
6: inflict {[}stun{]}

\subsection{Tags}\label{tags}

The ``tags'' are listed right after the name listed. This is where
mechanically-significant descriptors are stored. Some skills may require
weapons with particular tags, like ``range: close'' or ``hilt''. Weapon
tags describe how the weapon is handled. There are six main weapon tags,
``hilt'', ``pole'', ``one-handed'' (or ``1-h''), ``two-handed''
(``2-h''), reach and range.

\textbf{hilt}: A weapon with a hilt. A ``hilt'' is a ``well-defined''
handle or grip which often2 includes some kind of guard to protect the
users hand. Swords, knives, and daggers, are hilt weapons, as are patas
and katars.

\textbf{pole}: Pole weapons. Weapons with a long shaft rather than a
well defined grip. Pole weapons include spears, axes, hammers, and of
course, all polearms.

\textbf{one-handed}: A weapon designed to be used with one hand. If you
use two hands on a one-handed weapon, treat your STR as 2 points higher
for the purposes of meeting a STR requirement.

\textbf{two-handed}: If you try to wield a two-handed weapon with one
hand, attack with disadvantage.

\textbf{reach}: How many spaces does your weapon reach? If this tag is
absent, assume the weapon has reach: 1. If a weapon has {[}reach:
close{]}, you are treated as if you move briefly into the opponents
space to attack them. This is often relevant for particular skills.

\textbf{range}: Used for ranged weapons.

Tags can be combined to accommodate unusual weapons like
``hand-and-a-half'' a.k.a. ``bastard'' swords or
\href{https://www.youtube.com/watch?v=rHWGp5lGUNY}{this}
\href{https://swordsantiqueweapons.com/s1866_full.html}{thing}.

\subsection{Requirements}\label{requirements}

Not satisfying a weapon's requirements causes you to attack with
disadvantage.

\subsection{Speed}\label{speed}

The ``Speed'' of a weapon is the AP cost of an attack with the weapon.
If the weapon has multiple costs separated by arrows, you pay the first
cost on your first attack with the weapon in a round, the second cost on
your second, etc. If you are on the last listed cost, pay that cost
again to attack again. Attacking three times in one round with a weapon
with the cost ``1-\textgreater2'' costs 5 AP.

\subsection{To-Hit}\label{to-hit}

The score you add to an attack check with this weapon.

\subsection{Attacks}\label{attacks}

\subsubsection{Damage Types}\label{damage-types}

Attacks also have with damage types. There are three major types of
damage. They are: B for ``Bludgeoning'', P for ``Piercing'', and S for
``Slashing''.

There are also sub-types like Severing, Slicing, and Picking. Severing
damage counts as S, but S does not count as Severing. The damage
sub-type will share the first letter of the generic damage type. This
has worked so far because there are a lot of conveniently placed words
in English.

The most common sub-types of damage are:

\textbf{Slicing}: What a saber does. A long fast drawing cut, often with
a curved blade.

\textbf{Severing}: What an axe does. A chopping motion.

\textbf{Picking}: What a warpick does. A swinging motion.

\textbf{Bashing}: Reserved for heavy, crushing blows.

\subsubsection{Effects and Conditions}\label{effects-and-conditions}

Weapons often have extra effects that trigger ``on 6'', ``on matching''.
Rolling a 6 and a 4 will give you a total of 10 (+ any bonuses) and if
you succeed, the 6 might activate some additional effect. This makes
gaining advantage even more useful, as it increases your chance of
activating effects.

Often, weapons use common effects and inflict common statuses or
conditions. These conditions and effects are listed below. Conditions
are notated in brackets, effects are in quotes. Common weapon conditions
and effects are listed below.

\paragraph{Effects}\label{effects}

\textbf{``Bypass''}: Ignore an amount of damage reduction from armor
equal to your To-Hit bonus.

\textbf{``Pull''}: The target can be moved one space in any direction
except backwards. Target can spend 1 AP to trigger a STR contest. If
they win, they are not moved.

\textbf{``Push''}: The target is pushed back one space. Target can spend
1 AP to trigger a STR contest. If they win, they are not pushed back.

\textbf{``Rend''}: If armor reduced the damage of this attack note
whether the damage type was B, P, or S. Reduce the armor's damage
reduction against that damage type by the To-Hit bonus of this attack.

Common status effects are listed in their own section.

\paragraph{Combat Maneuvers}\label{combat-maneuvers}

These effects are inflicted in addition to damage on a successful
attack, they allow you to build a strategy around a certain fighting
style or weapon type. These effects and conditions are not meant to
encapsulate everything you can do! For everything else:

On a successful attack, describe the maneuver you are attempting to
perform. The target can either let you do it or take the damage of the
attack as usual.
\href{https://oddskullblog.wordpress.com/2021/11/15/combat-maneuvers-the-easy-way/}{Credit.}

Use this to disarm, knock people out, etc. You can also use this to do
what would normally be done by ``Push'' or some other effect, you just
won't get damage on top of the maneuver.

\section{Resting, Healing, Dying}\label{resting-healing-dying}

\subsection{Resting and Healing}\label{resting-and-healing}

A ``Rest'' is a period of light, relaxing, non-strenuous activity.
Resting includes, but is not limited to, eating, drinking, and sleeping.

If you rest for an hour, either: - Regain all standard health - Regain 1
Crit Health.

If you rest for 8 hours: - Remove 3 {[}exhaustion{]} and remove 1
{[}injury{]}.

\subsection{Dying}\label{dying}

If you reach 0 crit health, fall unconscious. All damage taken while you
are down is crit damage, if your negative crit health exceeds your total
crit health, you die. If your crit health becomes positive, you are
conscious again.

At the beginning of every round in which you are down, the DM rolls for
your character on an Individual Consequence Table. You do not know what
is on the table, and you do not know what you rolled until:

\begin{itemize}
\tightlist
\item
  You regain consciousness.
\item
  Someone checks on your body.
\item
  Someone attempts to administer aid to you.
\end{itemize}

The {[}stabilized{]} condition is a common consequence. The DM keeps
rolling every round until either you are {[}stabilized{]} or you die.
Death is another. When you become {[}stabilized{]} you stop rolling on
the consequence table and start Resting. If you take damage while
{[}stabilized{]}, you are no longer Resting, start rolling on the
consequence table again. At some point, the GM might stop rolling for
your character. Hopefully you're {[}stabilized{]}!

If the whole party is down, the GM rolls on a Party Consequence table.
This is almost certain death. You never know though, maybe you get
rescued/jailed/possessed/repurposed as a nutrient dense mulch.

\subsubsection{Consequence Tables}\label{consequence-tables}

There are three basic outcomes on a consequence table:

\begin{itemize}
\tightlist
\item
  You die.
\item
  Something bad happens, you recieve a debuff of some kind.
\item
  You become {[}stabilized{]}.
\end{itemize}

A ratio of 1:2:3 for {[}stabilized{]} / die / ``something bad happens''
is tough but fair. This is easily handled by a table with 6 entries:

\begin{itemize}
\tightlist
\item
  1-2: Death.
\item
  3-5: Something bad happens.
\item
  6: {[}stabilize{]}
\end{itemize}

If you want to be more lenient you can change the ratios or rule that a
character doesn't have to roll on the first round that they are out. Get
spicy, put something positive on the table, the world is your oyster.

\paragraph{Bad Things}\label{bad-things}

\subparagraph{Temporary Debuffs}\label{temporary-debuffs}

\begin{itemize}
\tightlist
\item
  {[}lame{]} for an hour.
\item
  Take 1 {[}exhaustion{]}.
\item
  {[}injury: disable an arm{]}.
\item
  Die if your crit health hits 0 again before your next rest.
\end{itemize}

\subparagraph{Permananent Debuffs}\label{permananent-debuffs}

\begin{itemize}
\tightlist
\item
  Lose an eye. Your max PER becomes 1.
\item
  Lower STR by 1.
\item
  Lower max crit health by 1.
\item
  Lose an arm.
\end{itemize}

Many groups don't like permanent debuffs! Make tables tailored to
whatever sort of game you are running. Maybe replace one Death outcome
with a permanent debuff, so you have 1 {[}stabilized{]}, 1 Death, 3
temporary debuffs, and 1 permanent debuff.

The role of these tables is to increase tension by introducing a high
degree of risk and uncertainty without \emph{necessarily} having to
increase the probability of character death.

\subsubsection{Sample Consequence
Tables}\label{sample-consequence-tables}

\paragraph{Individual Consequence
Tables}\label{individual-consequence-tables}

\subparagraph{Nothing d4}\label{nothing-d4}

\begin{enumerate}
\def\labelenumi{\arabic{enumi}.}
\tightlist
\item
  Death.
\item
  Nothing happens.
\item
  Nothing happens.
\item
  {[}stabilized{]}
\end{enumerate}

\subparagraph{Generic d6}\label{generic-d6}

\begin{enumerate}
\def\labelenumi{\arabic{enumi}.}
\tightlist
\item
  Death.
\item
  Death.
\item
  Die if you are reduced to (0) before your next rest.
\item
  Take 1 {[}exhaustion{]}.
\item
  {[}injury: disable an arm{]}.
\item
  You are {[}stabilized{]}.
\end{enumerate}

\paragraph{Party Consequence Tables}\label{party-consequence-tables}

In most cases, the thing that makes most sense is that the whole party
dies, but maybe if they are attacked while traveling, something else
happens!

Things that are context specific will be italicized.

\begin{enumerate}
\def\labelenumi{\arabic{enumi}.}
\tightlist
\item
  Everyone dies.
\item
  Everyone dies.
\item
  Each character flips a coin. On heads, they are {[}stabilized{]}, on
  tails they die.
\item
  The party wakes up in cells in \emph{local jail} with all of their
  things missing.
\item
  Miraculously, no one is dead. All of your things have been stolen.
\item
  The players wake up having been rescued by \emph{strange legendary
  creatures / local band of Merry Men.}
\end{enumerate}

\section{Common Status Conditions}\label{common-status-conditions}

\textbf{{[}bleed{]}}: Take 1 damage each turn until {[}bleed{]} is
removed. {[}bleed{]} can be removed by spending 3 AP fashioning and
applying a bandage or spending 1 AP applying a prepared bandage

\textbf{{[}exhaustion{]}}: Temporarily lower all of your scores by 1 for
every stack of {[}exhaustion{]} you have. Once none of your scores are
above 0, start lowering your movement speed by 1 instead. Once your
movement speed hits 0, fall {[}unconscious{]}. When you Rest, remove 1
{[}exhaustion{]}.

\textbf{{[}injury{]}}: The target chooses to either lower their move
speed by 1 by taking an injury to a leg, orthey take disadvantage while
using one limb. {[}injury{]} can be removed by resting while missing no
crit health, then flipping a coin and getting heads.

\textbf{{[}lame{]}}: Lower movement speed by 1.

\textbf{{[}poison{]}}: At the beginning of the round, take damage equal
to the number of {[}poison{]} you have. Remove 1 {[}poison{]}.

\textbf{{[}prone{]}}: Character is knocked to the ground and
{[}vulnerable{]} (so attacks against them have disadvantage). Getting up
costs 1 AP.

\textbf{{[}stun{]}}: When the character receives AP, discard it and
remove this condition.

\textbf{{[}vulnerable{]}}: If a character is knocked down, distracted,
being attacked from opposing directions, or otherwise in a
disadvantageous position, checks made against the character have
advantage.

\textbf{{[}stabilized{]}}: Do not roll on consequence tables. Begin
resting. If you take damage remove this condition.

\section{CASH}\label{cash}

CASH stands for Currency Abstraction System. The ``H'' is just an ``H''.
It is an optional ruleset for dealing with money that allows it to stay
in the fiction while also getting rid of bookkeeping. Not appropriate
for every game, but a good way to keep things moving quickly.

CASH expresses the narrative cost or value of something through broad
characterizations. Every character has a CASH level. CASH levels are
ordinal but not cardinal. Having a CASH level is like having a position
in a footrace. If you know you are in third place, you know your
relative position but not your actual distance from any other
participant. Your position is only defined relative to other positions.

\subsection{The Rules}\label{the-rules}

\begin{enumerate}
\def\labelenumi{\arabic{enumi}.}
\item
  You cannot purchase anything valued above your current CASH level.
\item
  If you purchase something valued at your current CASH level, lower
  your CASH level by 1.
\item
  If you purchase something 1 level below your current CASH level,
  perform a BARTER or SOCIAL check. If you fail the check, lower your
  current level of CASH by 1.
\item
  You can buy anything 2 or more levels below your current CASH level.
\end{enumerate}

\subsection{CASH Levels}\label{cash-levels}

\begin{enumerate}
\def\labelenumi{\arabic{enumi}.}
\tightlist
\item
  \textbf{Deprivation}: Food barely fit for human consumption. Most
  things you could purchase at this level are practically
  indistinguishable from trash.
\item
  \textbf{Destitution}: Some street food, maybe a loaf of bread. Rags.
  Ale.
\item
  \textbf{Indigence}: A cup of stew. Hardtack. Raggedy threadbare
  clothing. A night in a run-down shack. Maybe a club.
\item
  \textbf{Scarcity}: A short stay in a shared room. A hot meal.
\item
  \textbf{Penury}: A knife. An article of common clothing.
\item
  \textbf{Poverty}: A dagger. A backpack.
\item
  \textbf{Frugality}: A short stay in an inn. A cheap spear or a
  woodcutting axe. An old rickety open cart. A wooden shield. Rope.
\item
  \textbf{Austerity}: An old mule. A new outfit. A basic weapon like an
  axe, a spear, or a short sword. Padded armor.
\item
  \textbf{Moderation}: A pack animal. Provisions for a journey. Nice
  warm clothing. Leather armor. A sword.
\item
  \textbf{Substance}: A horse. A polearm. Fine Clothing. A skiff.
\item
  \textbf{Prosperity}: Chain mail armor. A brigandine. A helmet or
  breastplate.
\item
  \textbf{Affluence}: A small sailboat. A war horse. Fine weapons of
  war.
\item
  \textbf{Abundance}: Plate Armor. Bodyguards and servants.
\item
  \textbf{Fortune}: A mansion. A ship capable of crossing an ocean.
\item
  \textbf{Wealth}: Businesses. Militias.
\item
  \textbf{Riches}: An estate. A Galleon. Marble statues.
\item
  \textbf{Opulence}: This level of wealth commands political power, it
  is difficult to even conceptualize. Fleets of ships. Entire
  industries.
\item
  \textbf{Incredible Fortune}: You command armies. You own banks. Your
  face is on the money.
\item
  \textbf{Unfathomable Wealth}: You determine the course of entire
  economies. Your wealth is your power, your power is your wealth.
\item
  \textbf{Limitless Riches}: This level of wealth exists only in legend.
  The wealth of gods. The wealth of an immortal dragon sleeping on a
  planetary core of pure gold.
\end{enumerate}

\section{Hexes}\label{hexes}

Hexes are the foundation of ``scrivening'', a magic system relying
entirely on written representations. Those who write hexes are
``scriveners''. It is the most prominent magic system in the upcoming
setting. This is not the only magic system that would be compatible with
Actlite, and it is not the only magic system that I plan on making.

Hex magic is a \textasciitilde low, \textasciitilde hard magic system.
It's not for everyone. It is built for the type of player that wants to
tool around directly with magic, finding clever or unexpected uses,
interactions, and effects.3 This system is an example of ``opting-in''
to complexity, which I've written about before.

\subsection{Lore}\label{lore}

Hexes are symbols used by scriviners to directly access, manipulate, and
utilize abstract concepts. By manipulating these symbols, scriveners
cause tangible effects in the world around them. Hexes can be subjected
to any number of alterations, transformations, and modifications. Some
of these operations don't fundamentally change how the hex works, some
do. Different scriveners tend to write hexes in slightly different ways,
but they are all accessing the same concepts.

Scrivening uses a combination of written symbols and cognition.
{[}independent{]} hexes can handle cognition on their own, the scrivener
handles the cognition for any other hex. Making a hex {[}independent{]}
is difficult and requires a lot of modifications and transformations.
Because of this, studying an {[}independent{]} example of a hex is a
terrible way to try to learn the hex.

\subsection{Learning Hexes}\label{learning-hexes}

The easiest way to learn a particular hex is to study it along with some
explanation or guidance from the author of the hex. This is one of the
many reasons scrivening is a fractured and secretive art. The primary
method of teaching someone a particular hex involves giving them access
to your hexes directly, putting yourself at risk.

The typical way to learn hexes is to discover them for yourself using
the methods or protocols of the relevant school or tradition. Sometimes
this means feverish study, sometimes this means clear-headed meditation.

\subsubsection{Mechanics}\label{mechanics}

You can try to learn individual hexes by studying the hexes themselves
or you can roll on a hex archetype table. This magic system does not
hand players a great big catalog of spells to choose from. Every
character should end up with a unique set of hexes that they have to
learn to utilize effectively. You are meant to improvise with what you
get and use your hex library to write your own signature combinations.

\paragraph{Learning Combined Hexes}\label{learning-combined-hexes}

When you succeed in learning a hex that is composed of multiple hexes,
you learn every unknown hex in the combination. The DC to learn a
combined hex is +3 for every hex in the combination.

Studying a hex combination takes an hour per hex. You can only attempt
to learn a given hex once per day. You cannot study {[}independent{]}
hexes.

\subsection{Hex Activation}\label{hex-activation}

When a hex is destroyed it is activated. When it is activated, it is
destroyed. If it is activated while some effect requirement is not met,
it is destroyed. ``Destroyed'' includes smudged to illegibility, the
medium that it is written on is torn, etc.

\subsection{Writing and Casting Hexes}\label{writing-and-casting-hexes}

A character can write 3 + WIL hexes, this capacity is reset after every
rest. A character can maintain 3 + WIL hexes at a time. If a character
writes another hex while they are maintaining as many as they can, the
olest hex is destroyed.

Each hex can be modified by up to 3 + WIL {[}modifier{]} hexes.

Once written, a hex remains usable until you sleep or until 24 hours
pass, whichever happens first.

To write a hex during initiative you have to spend 3 AP for the hex and
1 AP for every modifier.

Many hexes have some kind of activation condition. Often this condition
is some action the character might take, like attacking. In these cases
you don't have to pay AP to use the hex. If you want to search for an
already written hex you do not already have in your hand and then
activate it, pay 1 AP.

\subsubsection{{[}modifier{]} Hexes}\label{modifier-hexes}

{[}modifier{]} hexes can be combined with other hexes. If one hex in the
combination is activated, all are. {[}modifier{]} hexes do not affect
individual hexes within a combination, they affect the whole
combination.

\subsubsection{{[}independent{]} Hexes}\label{independent-hexes}

Independent hexes do not lose power over time and are not bound to their
author. As a result, effects that target the author do not work.

\subsection{Additional Context}\label{additional-context}

Individual hexes might override any rule listed above. For example, many
{[}modifier{]} hexes can be added to hexes you did not write.

Hexes can also impose additional restrictions, requirements, or costs.
For example, taking extra time to create, having unique costs for
creation or activation, or only being compatible with certain hexes.

\subsubsection{{[}durable{]} Casting and Hex
Tables}\label{durable-casting-and-hex-tables}

{[}durable{]} hexes can be written into a ``hex pallete''. Hexes can be
built from combinations of hexes in the same table and immediately cast
without the caster having to write the hex down. Diegetically, a hex
``table'' might be called a pallete, a matrix, or something else
depending on the tradition or school.

{[}durable{]} hexes often come with additional or alternative casting
requirements. Some hexes can only be cast as {[}durable{]} hexes.
{[}independent{]} Hexes

To make a hex {[}independent{]}, a scrivener must attach a
{[}modifier{]} that consists of two parts.

The first is the ``key''. The ``key'' is the first step in breaking the
unique connection between scrivener and hex. It varies from person to
person and from hex to hex. The writer must figure out the unique key
for each hex they cast, they cannot use someone else's.

The second part is a ``hex signature''. This is a unique hex that you
will use every time you make a hex {[}independent{]}. The signature does
not, on its own, identify you. if someone sees your signature they won't
necessarily know it is yours. However once they do know, they will
recognize the signature in every {[}independent{]} hex you write. You
cannot change your signature.

Two people writing the same non-{[}independent{]} hex will write nearly
identical hexes. Once they make their hexes {[}independent{]} they have
to leave distinct signatures. Unless steps are taken to obfuscate the
contents of the entire hex, the signature is obvious. The process of
making a hex {[}independent{]} often means adding a large number of
alterations that make it hard for other scriveners to learn the hex, but
reading it remains relatively easy.

Some hexes can be made {[}independent{]} relatively easily. For some
hexes the process is very difficult or costly. Many hexes (especially
powerful hexes) cannot be made independent through any known means. If a
hex is not {[}independent{]}, it is bound to an author and follows the
usual casting rules

\subsubsection{Invocations}\label{invocations}

Invocations are {[}independent{]} hexes written on long and narrow
strips of material. Most often this is paper. Invocations are by far the
most common type hex. Many people never come into contact with hexes,
most that do only come in contact with invocations. Invocations
typically have relatively minor effects. Invocations are almost always
activated by tearing the medium they are written on.

Invocations are almost always {[}independent{]} so that they do not lose
potency. Additionally, {[}modifier{]} hexes that harm the author are
very easy to write, so any sane scrivener wouldn't part with a hex
without making it {[}independent{]} first.

Some of the reasons for the format of the invocation is practical. A
long slip of paper is easy to fully rip in half, guaranteeing
activation. They are easy to tear with one hand. This is especially
useful for weapon invocations, where the invocation is wrapped around
either the handle or the hand wielding the weapon. Prayers are written
in a similar format.

\section{Original Posts}\label{original-posts}

A lot of the first drafts for these rules appeared on the
\href{https://actlite.substack.com/}{Substack}. The Substack posts often
include optional rules, content, and more context around why certain
mechanics were written they way they were. They also often include
outdated rules.

I'll link those posts here in roughly the order they appeared here.

\begin{itemize}
\tightlist
\item
  \href{https://actlite.substack.com/p/core-rules}{Core Rules}
\item
  \href{https://actlite.substack.com/p/weapons}{Weapons}
\item
  \href{https://actlite.substack.com/p/resting-healing-dying}{Resting,
  Healing, Dying}
\item
  \href{https://actlite.substack.com/p/the-essential-johnny-cash}{CASH}
\item
  \href{https://actlite.substack.com/p/hexes-part-1-the-rules}{Hexes
  Part 1}
\item
  \href{https://actlite.substack.com/publish/posts/detail/142442758?referrer=\%2Fpublish\%2Fposts}{Hexes
  Part 2}
\end{itemize}



\end{document}
