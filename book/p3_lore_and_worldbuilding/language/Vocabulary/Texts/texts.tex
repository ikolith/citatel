\documentclass{article}
\usepackage{multicol}
\usepackage{enumitem}
\usepackage{tipa}
\setlist[enumerate]{noitemsep}

\begin{document}

\tableofcontents
\newpage

\section{Babel}

\begin{multicols}{2}

\begin{enumerate}
\item Now the whole world had one language and a common speech.  
\item As people moved eastward, they found a plain in Shinar and settled there.


\item They said to each other, ``Come, let’s make bricks and bake them thoroughly.'' 
\item They used brick instead of stone, and tar for mortar.  
\item Then they said, ``Come, let us build ourselves a city, with a tower that reaches to the heavens, so that we may make a name for ourselves; otherwise we will be scattered over the face of the whole earth.''

%"brick.ACC.PL  make.OPT.1P.PL intensifier-cook.OPT.1P.PL" say.3P to-themselves

%the.c2 bricks.INST  conj.AS the.c3 stone.INST, the.c4 tar.INST conj.AS the. c2 mortar.INST.

%"the.C2.SG city.ACC.SG with.PART tower.LOC sky.ALL.C1.  build.OPT.1P.PL.PRES, SO-THAT honor.ACC create.COND.FUT.1P.PL PRON.1P.PL" say.3P to-themselves



\item But the Lord came down to see the city and the tower the people were building. 
\item The Lord said, ``If as one people speaking the same language they have begun to do this, then nothing they plan to do will be impossible for them.  
\item Come, let us go down and confuse their language so they will not understand each other.''





 \item So the Lord scattered them from there over all the earth, and they stopped building the city. 
 \item That is why it was called Babel --- because there the Lord confused the language of the whole world. 
 \item From there the Lord scattered them over the face of the whole earth.


\end{enumerate}


\newcolumn

\begin{enumerate}
\item varet qa, ra ma\textipa{\.*n}aleje l\textipa{\=o}rala exkej a\textipa{\.*r}agav.
\item yi kis\textipa{\=i} re meraz a\textipa{\.*m}agav, pse hejarili Xinarit pe\textipa{\.*v}nem a quit maze\textipa{\.*l}enem.
\item ``ye ge\textipa{\.*b}olon kuram a manrojem'' axrizin se\textipa{\.*p}em.
\item xun muloron ge\textipa{\.*b}ojon, a yejder x\textipa{\=i}\textipa{\.*b}ij.
\item ``ya manomala a ya va\textipa{\.*s}ala xisez huerulu kuvem, qex osmiz kregala keram, hefe osmi ariafk\textipa{\=i}'' se\textipa{\.*p}em.
\item sema ra Sag\textipa{\=a}\textipa{\.*r}a \textipa{\=u}\textipa{\.*r}ag qex ya manomala a ya va\textipa{\.*s}ala ak\textipa{\=a}va kisirin li\textipa{\.*g}.
\item ``fe aki\textipa{\.*s}ir exk a l\textipa{\=o}rar exk isili keovevem, ma\textipa{\.*n}a aqiadi vi oyexi''
\item ``telajez vam a l\textipa{\=o}rala rupsam, qex axrilin hevom qevavam''
\item ra Sag\textipa{\=a}\textipa{\.*r}a yin kisilin qix ma\textipa{\.*n}alejz re\textipa{\.*f}, a v\textipa{\=a}va manomaj ape\textipa{\.*x}.
\item qir ``Babel'' qa manoma aexpaban --- vle qit ra Sag\textipa{\=a}\textipa{\.*r}a l\textipa{\=o}ralan ma\textipa{\.*n}alejej mane\textipa{\.*f}.
\item qix ra Sag\textipa{\=a}a\textipa{\.*r}a ra ma\textipa{\.*n}alejez yin kisilin re\textipa{\.*f}. 

\end{enumerate}

\end{multicols}

\section{The Fable of the Bog Flower}

%\begin{multicols}{2}

\begin{enumerate}

\item Long ago, the god Varīg, the Slimy One, whose home is the Chthonic Pond, was tending to their garden, which is the Great Swamp. 
\item As they did this, they saw a child limp down the path to the Holy Grove and, using a heirakar\footnote{A dagger used for certain rituals} with great respect and care, take a small cutting of the white bog flower.
 \item Varīg was impressed by the child’s piety, and blessed the cutting. 
 \item When the child returned to their village, they brewed the flower into bitter tea and drank it. 
 \item The next morning, the child awoke to find their limp was cured.  
 \item The adults of the village, seeing the child’s blessing, grew greedy; they thought of using the bog flower to grow stronger and destroy their enemies. 
 \item They ran down the path to the Holy Grove heedlessly, and tore up the white flower by the handful. 
 \item Then they stuffed it whole into their mouths, not taking care to brew the tea in the old way. 
 \item Varīg saw this heedlessness and cursed them for their greed.
 \item  All who ate of the flower that day were wracked with great pain, many died, and those who did not were crippled until the end of their days. 
 \item This is why, to this day, the bog flower is called the Varīgoxvlor\footnote{Ambiguously translatable as  \textit{Varig
 s bloom} or \textit{The flower that can cure or wound}} end{enumerate}



%\end{enumerate}

\end{multicols}
\end{document}