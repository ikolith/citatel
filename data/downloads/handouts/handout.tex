% TODO: this is all super outdated and i dont know how to deal with universal combat skills tbh

\documentclass{article}


\usepackage{multicol}
\usepackage[ margin= .5in]{geometry}
\usepackage{enumitem}

\usepackage[most]{tcolorbox}


\usepackage[thinlines]{easytable}

\begin{document}

\thispagestyle{empty}
%\setlist{noitemsep}
\setlength{\columnsep}{.1cm}

%  \begin{minipage}{\linewidth}{}
\begin{multicols}{2}
  \begin{itemize}
    \item \textbf{Basic Attack [interrupt]:} Any move tagged with [BA] is a Basic Attack.
    \item \textbf{Block:} Get the armor bonus from your shield until your next turn. In-game you are assumed to be turning around to block attacks as they come in. You can choose not to block an attack. If two attacks happen at the same time, attacking different sides of your character, you can only block one.
    \item  \textbf{Called Shot:} ``Called Shot'' allows you to take an attack that attempts to achieve some specific goal not covered by another skill, like hitting a weak spot. Using ``Called Shot'' increases the cost to perform an action (typically [BA]) by 1 AP. Take a -3 to hit (or more depending on what you’re doing, TTYDM) and either do +2 more damage or achieve whatever goal you had in addition to doing damage.
    \item \textbf{Dodge [interrupt]:} [2n1] add 2 + AGI to your To-Hit until your next turn.
    \item \textbf{Pommel/Haft Strike:} [1n1]. A [BA] using the pommel or haft of a weapon. STR to hit, 1d4 (B) damage.
    \item \textbf{Move:} [1n1]. Move a distance equal to your move speed.
    \item \textbf{Sneak Attack 0:} When attacking an enemy who is unaware of you, roll double damage dice (before any other ``Stealth'' effects take place).
    \item   \textbf{Wait:} You cannot do anything on the same turn that you ``Wait''. Regain 1 AP at the end of your turn.
  \end{itemize}



\end{multicols}
%\end{minipage}




\smallskip
\begin{tcolorbox}[breakable, enhanced]
  The cost in SP of increasing your score from X to Y is Y. 2 STR to 3 STR costs 3 Skill points. When your character has some amount of the Healing status effect, on your turn, you first check how much heal you have, increase your health by that much, and then reduce the amount of heal you have by 1. Many things in the game (like poison damage) are like this, they increase or decrease by 1 every turn/step.
\end{tcolorbox}

\begin{multicols}{2}
  \begin{itemize}

    \item When attacking [vulnerable] enemies, roll an extra d6 to hit and roll double damage dice.
    \item The [frail] status effect makes one take double damage from poison and bleeding, and lowers max CON by 2 until no longer [frail].
    \item Prone, grappled, unaware, distracted, blinded, or otherwise disadvantaged enemies are [vulnerable].
    \item You can ignore CON stacks of a status effect, e.g. if you have 1 CON and 1 stack of poison and 2 stacks of bleeding, you only take damage from one stack of bleeding. Many status effects are tracked on the AP tracker.
    \item You can decide to take 10 minutes to do something (within reason) and treat your check as a 10.
    \item You get 1d6 hit die per 5 health, it takes an hour of rest to roll a hit die.
    \item Any adjacent characters in initiative order can choose to take their turns at the same time, doing actions in whatever order they want. You can hold your action to go down in initiative order for any reason, including being adjacent to a teammate.
    \item Requirements are inherited. If some ``Skill 2'' requires ``Skill 1'' then ``Skill 2'' also requires everything that ``Skill 1'' required, and if you don’t fit those requirements, you can’t use the skill.
    \item Passive scores are 6 + SCORE.
    \item Fall damage is a d6 for every 6 spaces (up to 25d6).

  \end{itemize}
\end{multicols}
\begin{itemize}
  \item \textbf{Interrupt:}
        At any time outside of your turn, you can choose to interrupt another (PC or NPC) character's action. You cannot interrupt if they have already rolled the check associated with the action. You cannot interrupt an interrupt. You cannot use interrupt to move out of an attack's range. Roll a contest, your action against theirs. The higher check goes first, the interrupted opponent can choose not to take the action they were going to take. If you held your turn, or your turn has passed but you have some AP to spend, you can use [interrupt] without paying any AP. Otherwise, the interrupt attempt costs [1n2].
  \item \textbf{Grappling:} Entering grappling [3n1], +AGI to hit, followed by a STR contest. If the party entering the grapple wins, both parties are [grappling], but the defender, the one who was grappled, is [vulnerable]. Defender can spend [3n1] to trigger another STR contest to try and break free.
\end{itemize}

%name, HP, to-hit
