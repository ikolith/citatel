\documentclass{article}
\usepackage{tipa}
\newcommand{\ti}{\textipa}
\begin{document}

Loric's number system is base--12 or dozenal. The basic cardinal number words are:
\bigskip

\begin{center}
\begin{tabular}{|c  c || c  c || c  c|}
\hline
  1 & exk & 13 & exkema & 36 & xu\ti{\.*f}ir \\
  2 & a\ti{\.*l}a & 14 & a\ti{\.*l}ama & 48 & ro\ti{\.*m}iye\\
  3 & qe & 15 & qexever & 60 & xeveye \\
  4 & rom & 16 & ma\ti{\.*r}oma & 72 & xe\ti{\.*f}e\\
  5 & xeve & 17 & xevema & 84 & me\ti{\.*z}iye\\
  6 & xef & 18 & xe\ti{\.*f\=e} & 96 & nu\ti{\.*r}iye\\
  7 & mez & 19 & mezema & 108 & qeqe\ti{\.*r}iye\\
  8 & nure & 20 & nurema & 120 & xe\ti{\.*l}eye\\
  9 & qeqer & 21 & qemezer & 132 & he\ti{\.*m}e\\
  10 & xe\ti{\.*l}e & 22 & xe\ti{\.*l}ama & 144 & o\ti{\.*m}\\
  11 & hem & 23 & hemaa\ti{\.*la} & 1728 & qema\ti{\.*n}a\\
  12 & ma & 24 & m\ti{\=a\.*l}a & 20736 & o\ti{\.*m}or\\
\hline
\end{tabular}
\end{center}
\medskip

\iffalse
%For the most part the pattern is clear, with some exceptions:
\begin{itemize}

  \item
  \item Odd multiples of 3 that are smaller than 24 are constructed as a compound word literally meaning ``three-by-[number],'' e.g. \textit{qexever} ``fifteen'' is broken down into \textit{qe-xeve-r}, with the instrumental case ending \textit{--r} indicated multiplication in this context. Above 24 (\textit{manxa\ti{\.*l}a}) this pattern breaks this pattern breaks;

\end{itemize}
\fi

\end{document}
