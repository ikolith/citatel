\documentclass{article}
\usepackage{tipa}
\begin{document}
Loric, or endogenously \textit{L\={o}ra} ``the spoken thing, the word,'' is the collective name for all extant members of the Loric language family.  Most commonly the name is used to refer only to the two Loric languages with sizable written corpora that serve as prestige forms, those being Old High Loric (\textit{L\={o}ra S\={a}garaj}, ``speech of kings'') and Middle Loric (\textit{L\={o}ra D\={o}ma}, ``well-ordered speech, civilized speech''). Excluding these two varieties, the remaining dialect continuum is sometimes collectively referred to as Low Loric (\textit{L\={o}ra Mans\={e}lj}, literally ``over-salted speech''), but this apparent unity is misleading; the Low Loric languages exhibit endless variety, and the vernacular of one village may be totally unintellig\-ible to a village only a few miles away.

In general, the ``higher'' forms of Loric are more conservative, and have retained more features of  the now extinct ancestor language to the family, Proto-Loric. Old High Loric exhibits an extensive case system with varying levels of formality, as well as a fourfold number class system (singular, dual, paucal, plural.) Middle Loric is still highly inflected, but has reduced the number system to the more familiar singular/plural distinction for verbs and a threefold singular/paucal/plural distinction for most nouns. Formality has also been simplified from a morphological requirement to a set of free particles. No comprehensive grammar of any Low Loric language exists, since the very people with the expertise to write it are the ones who look down on nonstandard variation as ``over-salted,'' but some information can be reconstructed from the complaints of urbane scholars. Overall, the Low Loric varieties seem to have moved almost completely away from inflection in favor of syntactic word order; pronouns, previously implied by verb morphology and thus not commonly used outside of poetry, have been re-lexicalized from kinship nouns. The formality system has more or less completely vanished, which may be a cause of the complaint that the people of the hinterland are anarchistic and lack respect for their supposed betters. However, as the academic establishment has a vested interest in promoting the more conservative elements as the one true way of speaking, a full survey of these languages will probably never come about.

Sociolinguistically, Old High Loric exists only as a liturgical language and language of court; while it may have a small number of native speakers in highly insular monastic environments, for the most part it is a second language learned by clergy and scholars. Middle Loric is the largest single language in the group by number of speakers, and serves as the lingua franca of commerce, politics, science, and literature. While the specific dialect used for this purpose contains a combination of elements not found in any specific regional dialect, it is mutually intelligible with the native language varieties of most urban or cosmopolitan centers. Middle Loric is the language described by this document.
\end{document}
